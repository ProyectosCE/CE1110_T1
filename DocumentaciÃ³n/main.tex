\documentclass[letter,12pt]{article}
\usepackage[left=2.54cm, right=2.54cm, top=2.54cm, bottom=2.54cm]{geometry} % Márgenes APA
\usepackage{setspace} % Interlineado
\setlength{\parindent}{1.27cm} % Sangría estándar APA
\setlength{\parskip}{0pt}     % Sin espacio entre párrafos
\usepackage{graphicx} % Para insertar imágenes
\usepackage{float} % Para posicionar imágenes y tablas
\usepackage{caption} % Para modificar captions de tablas e imágenes
\usepackage{csquotes} % Para mejor manejo de citas
\usepackage{amsmath} % Para ecuaciones matemáticas
\usepackage{svg}
\usepackage{indentfirst}

%==========REFERENCIAS=============
\usepackage[backend=bibtex,style=ieee,biblabel=dot,language=spanish]{biblatex}
\addbibresource{referencias.bib}

\usepackage[
    xetex,
    pdftitle={Tarea \# 1 - FFT y Sistemas de Modulación}, % Título del PDF
    pdfauthor={Barquero, J., Feng, J., Montero, A.},   % Autor del PDF
    pdfsubject={CE1110 - Análisis de Señales Mixtas}, % Tema del PDF
    pdfkeywords={}, % Palabras clave del PDF
    pdfproducer={LaTeX with hyperref package},
    pdfcreator={pdfLatex}
]{hyperref}

\hypersetup{
    colorlinks=true,        % Colorea los enlaces en lugar de usar cajas alrededor de ellos
    linkcolor = black,
    urlcolor  = blue,
    citecolor = black,
    anchorcolor = blue         % Color de los enlaces externos
}

\begin{document}
% ====== Portada ======
\begin{titlepage}
    \centering
    {\LARGE \textbf{Tarea \# 1} \par}
    {\LARGE FFT y Sistemas de Modulación \par}
    \vspace{1.5cm}
    {\large José Bernardo Barquero Bonilla \\ 2023150476 \par}
    {\large Jimmy Feng Feng \\ 2023060347 \par}
    {\large Alexander Montero Vargas \\ 2023166058 \par}
    \vspace{1.5cm}
    {\Large Instituto Tecnológico de Costa Rica \par}
    {\large Escuela de Ingeniería en Computadores \par}
    {\large Curso: CE1110 - Análisis de Señales Mixtas \par}
    \vspace{1.5cm}
    {\large Profesor: Luis Alberto Chavarría Zamora \par}
    \vfill
    {\large 15 de Octubre de 2025 \par}
\end{titlepage}

% ====== Desarrollo ======
\section{DFT/FFT y serie de Fourier}

La Transformada Discreta de Fourier (DFT) es la definición matemática que proyecta una secuencia finita \(x[n]\) de longitud \(N\) sobre exponenciales complejas y entrega \(N\) muestras espectrales \(X[k]\) igualmente espaciadas en frecuencia \cite{OppenheimSchaferDTSP3e,VetterliKovacevicGoyalFSP2014}. La DFT se calcula como
\[
X[k]=\sum_{n=0}^{N-1} x[n]\,e^{-j2\pi kn/N},\quad k=0,\dots,N-1,
\]
y la inversa como
\[
x[n]=\frac{1}{N}\sum_{k=0}^{N-1} X[k]\,e^{+j2\pi kn/N},\quad n=0,\dots,N-1.
\]
Estas expresiones fijan la convención de normalización usada a lo largo del documento \cite{OppenheimSchaferDTSP3e}.

La Transformada Rápida de Fourier (FFT) no es una transformada distinta, sino un algoritmo que evalúa la DFT con complejidad \(O(N\log N)\), en contraste con los \(O(N^2)\) de la evaluación directa \cite{CooleyTukey1965}. En la práctica, “hacer una FFT” significa calcular la DFT de forma eficiente; la distinción es: \emph{qué} se calcula (DFT) versus \emph{cómo} se calcula (FFT) \cite{OppenheimSchaferDTSP3e,NumPyFFT2024}.

Para una señal discreta \(x[n]\), la transformada \(X(e^{j\omega})\) (DTFT) es continua en \(\omega\) y periódica con período \(2\pi\) \cite{OppenheimSchaferDTSP3e,NISTDLMF_Fourier}:
\[
X(e^{j\omega})=\sum_{n=-\infty}^{\infty} x[n]\,e^{-j\omega n},\qquad
X(e^{j(\omega+2\pi)})=X(e^{j\omega}).
\]
La DFT puede interpretarse como un \emph{muestreo} de esta DTFT en \(N\) frecuencias uniformemente espaciadas \( \omega_k=2\pi k/N \), o en Hz \( f_k=\tfrac{k}{N}f_s \), con resolución \(\Delta f=\tfrac{f_s}{N}\) \cite{OppenheimSchaferDTSP3e,VetterliKovacevicGoyalFSP2014,SciPyFFT2024}. Aumentar \(N\) mejora la resolución; el \emph{zero-padding} densifica la grilla de lectura sin aumentar la resolución física \cite{SciPyFFT2024,NumPyFFT2024}.

La relación con la serie de Fourier surge de la periodicidad. La serie de Fourier representa una señal periódica continua \(x(t)\) como suma de armónicos discretos \(k f_0\) con coeficientes \(c_k\) \cite{OsgoodFTAMS2019,NISTDLMF_Fourier}:
\[
x(t)=\sum_{k=-\infty}^{\infty} c_k\,e^{j2\pi k f_0 t},\qquad
c_k=\frac{1}{T_0}\int_{0}^{T_0} x(t)\,e^{-j2\pi k f_0 t}\,dt.
\]
Análogamente, al tomar un bloque de \(N\) muestras y asumir su repetición periódica en tiempo discreto, la DFT entrega los coeficientes de la serie de Fourier de esa señal periódica discreta: tiempo (discreto) periódico \(\Rightarrow\) espectro (discreto) en \(N\) líneas \cite{OppenheimSchaferDTSP3e,VetterliKovacevicGoyalFSP2014}.

Finalmente, respecto al “muestreo del entorno continuo”, muestrear \(x(t)\) cada \(T_s\) genera \(x[n]=x(nT_s)\) y replica el espectro continuo \(X_c(f)\) cada \(f_s=1/T_s\); si \(f_s<2f_{\max}\) ocurre aliasing \cite{OppenheimSchaferDTSP3e,OsgoodFTAMS2019}. Sobre esta señal ya discreta, la DFT toma muestras de la DTFT en los bins \(f_k\), cerrando la cadena conceptual: muestreo temporal \(\Rightarrow\) réplicas en frecuencia; ventana temporal \(\Rightarrow\) suavizado por convolución; DFT/FFT \(\Rightarrow\) muestreo uniforme del espectro discreto-periódico \cite{OppenheimSchaferDTSP3e,SciPyFFT2024}.



% =========================
\section{Experimento 1: FFT sobre audio libre}
% Esta sección cubre la parte 2 (implementación FFT sobre audio libre) EN RESULTADOS,
% sin incluir código: solo describe datos, método, y muestra/analiza figuras.
\subsection{Datos y preparación}
% - Fuente del audio (licencia libre, citar en referencias).
% - Duración, tasa de muestreo, preprocesamiento (mono, normalización).
\subsubsection{Configuración de análisis}
% - Ventana usada (o sin ventana), N de la FFT, zero-padding (si se usó).
% - Justificación de elecciones (ruido/armónicos/estacionariedad).
\subsection{Resultados espectrales}
% - Insertar figuras de magnitud y fase (exportadas por el script).
\begin{figure}[H]
  \centering
  \includegraphics[width=0.95\linewidth]{figs/fft_magnitud_audio.png} % <-- reemplaza con tu PNG
  \caption{Espectro de magnitud $\lvert X(f)\rvert$ de la pista analizada.}
  \label{fig:fft-mag-audio}
\end{figure}
\begin{figure}[H]
  \centering
  \includegraphics[width=0.95\linewidth]{figs/fft_fase_audio.png} % <-- reemplaza con tu PNG
  \caption{Espectro de fase $\angle X(f)$ de la pista analizada.}
  \label{fig:fft-fase-audio}
\end{figure}
\subsubsection{Interpretación}
% - Identificar picos, armónicos, banda ocupada, ruido.
% - Comentar fase (linealidad, retardos, saltos 2\pi si los hay).
% - Relacionar con ventana y \Delta f. Señalar limitaciones.

% =========================
\section{Uso de Fourier en modulación y demodulación}
% Esta sección cubre la parte 3 (explicar a nivel de bloques cómo Fourier participa).
\subsection{Diagrama de bloques del sistema}
% - Incluir un diagrama: fuente -> modulador -> canal -> demodulador -> salida.
% - Puedes usar una figura SVG o TikZ exportada como PDF/PNG.
\begin{figure}[H]
  \centering
  \includegraphics[width=0.95\linewidth]{figs/diag_mod_demod.png} % <-- reemplaza con tu diagrama
  \caption{Diagrama de bloques del modulador-demodulador considerado.}
  \label{fig:bloques-mod-demod}
\end{figure}
\subsubsection{Señales esperadas en cada bloque}
% - Bocetos de formas de onda / espectros: banda base, portadora, bandas laterales.
% - Dónde se produce la traslación en frecuencia (mezcla), y dónde el filtrado.
\subsection{Lectura en el dominio de Fourier}
% - Explicar con palabras cómo se verían los espectros en cada etapa.
% - Vincular con filtros (PB/PBanda), sincronía y pérdidas por canal.

% =========================
\section{Bibliotecas y \emph{drivers} en microcontroladores}
% Esta sección cubre la parte 4 (bibliotecas MCU).
\subsection{Biblioteca/Driver}
% - Tabla con columnas: Bloque | MCU/Framework | Biblioteca/Driver | Comentario.
% - Ejemplos: CMSIS-DSP (ARM), ESP-DSP (ESP32), ArduinoFFT, KissFFT, ADC/DAC/I2S.

\subsubsection{Consideraciones de implementación}
% - Memoria (tamaño de buffers), tiempo real (latencias), precisión en punto fijo.
% - Consumo energético, reloj, estabilidad de osciladores.

% =========================
\section{Prototipo en PC: modulación y demodulación}
% Esta sección cubre la parte 5 (demo PC). Solo resultados e interpretación.
\subsection{Especificación del esquema de modulación}
% - Elegir uno (p.ej., AM-DSB, AM-SSB, BPSK, QPSK).
% - Parámetros: fs, fc, filtros (ancho de banda), niveles, SNR si hubo ruido simulado.
\subsubsection{Procedimiento experimental}
% - Flujo de generación: señal base -> modulador -> canal (opcional) -> demod.
% - Cómo generan figuras: script -> PNGs; lista de archivos producidos.
\subsection{Resultados en el tiempo}
% - Figuras de señal modulada y demodulada en el tiempo. Rotular ejes con unidades.
\begin{figure}[H]
  \centering
  \includegraphics[width=0.95\linewidth]{figs/mod_tiempo.png} % <-- reemplaza con tu PNG
  \caption{Señal modulada en el dominio del tiempo.}
  \label{fig:mod-tiempo}
\end{figure}
\begin{figure}[H]
  \centering
  \includegraphics[width=0.95\linewidth]{figs/demod_tiempo.png} % <-- reemplaza con tu PNG
  \caption{Señal demodulada (salida) en el dominio del tiempo.}
  \label{fig:demod-tiempo}
\end{figure}
\subsection{Resultados en frecuencia}
% - Espectros de la señal modulada y demodulada (magnitud; opcional fase).
\begin{figure}[H]
  \centering
  \includegraphics[width=0.95\linewidth]{figs/mod_frecuencia.png} % <-- reemplaza con tu PNG
  \caption{Espectro de la señal modulada.}
  \label{fig:mod-freq}
\end{figure}
\begin{figure}[H]
  \centering
  \includegraphics[width=0.95\linewidth]{figs/demod_frecuencia.png} % <-- reemplaza con tu PNG
  \caption{Espectro de la señal demodulada.}
  \label{fig:demod-freq}
\end{figure}
\subsubsection{Discusión de desempeño}
% - ¿Se recuperó correctamente la banda base? ¿Distorsión/ruido?
% - Efecto de filtros y de sincronía (portadora/fases). Limitaciones del prototipo.


    
% ====== Referencias ======
\newpage
\printbibliography

\end{document}